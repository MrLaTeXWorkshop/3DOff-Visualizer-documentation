\section{Introdução}

Esse programa feito para a disciplina de Computação gráfica tem como objetivo integrar
C++, Qt e OpenGL para fazer uma aplicação que seja capaz de abrir arquivos .off (Object File Format)
e manipulá-los. O programa, em grande parte, consiste na implementação do artigo 
\emph{"Interactive Graphics Applications with OpenGL Shading Language and Qt"} com apenas algumas 
melhoras a sua interface e a adição da leitura de malha mista como uma nova funcionalidade.

\section{Organização do código}

Utilizamos C++ como a linguagem de programação, o Qt como o framework para 
construir a interface visual no qual o usuário interage e o OpenGL é a API gráfica que renderiza
os objetos, processa shaders e texturas. As seções a seguir serão separadas da seguinte forma: primeiro
apresentaremos a comunicação entre a interface visual do Qt e o código para em seguida apresentarmos o código 
em si e sua estruturação.