\section{Código}

Agora que a comunicação na interface do Qt foi detalhada,
podemos passar a explicar a comunicação dentro do código em si. Como dito 
anteriormente, o maior componente dentro do projeto que possui a grande porção
do código é o QGLWidget. Ademais, temos a \emph{main.cpp}, que inicia a aplicação e 
inclusive inicia a \emph{mainwindow.cpp}, que é a janela principal, onde está o QGLWidget.

\subsection{glwidget.cpp}

Esse é o código referente a tela do QGLWidget, encapsulando todas as operações 
referentes ao OpenGL. Além de tratar operações recebidas do usuário na interface. 
O glwidget é o arquivo mais importante, pois além de ser responsável em comunicar 
com a API gráfica, todas as outras classes são orquestradas a partir dessa: 
a \emph{camera.cpp} a \emph{light.cpp}, o \emph{material.cpp} e o \emph{trackball.cpp}.

Neste momento não seria intuitivo falar sobre todas as funções, pois além de estarem comentadas, 
entraremos em mais detalhes sobre o funcionamento das principais funções quando falarmos 
sobre os modelos matemáticos. Então, agora iremos completar a discussão sobre a interação do 
Qt, discorrendo como o código faz o tratamento dos pedidos do usuário.

\subsubsection{Nova cor para o fundo do QGLWidget}

Quando o usuário pressiona para escolher uma nova cor para o plano de fundo do visualizador, 
a função \emph{chooseBackgroundColor()} é chamada. Ela é responsável em abrir o \emph{QColorDialog},
um Widget já implementado dentro da framework, que possui a função \emph{getColor()} permitindo chamar 
a paleta de cores para que o usuário faça uma escolha. Logo após isso, o programa apenas valida a cor 
e se for válida ela será definida como a nova cor de fundo.

\subsubsection{Gerenciamento de arquivos}

Ao pedir para abrir um novo arquivo, a função \emph{showFileOpenDialog()} é chamada para fazer essa tarefa.
Esta função pode ser divida em duas partes: a primeira é abrir o gerenciador de arquivos, usando o 
QFileDialog, para deixar o usuário escolher um objeto. Caso um objeto válido tenha sido escolhido, 
o programa então agora irá ler e processar o arquivo para gerar o objeto na tela.

Essa lógica é aplicada da mesma maneira quando é salvo uma imagem do frame atual do OpenGL usando a 
função \emph{takeScreenshot()}. Nesta função, o QFileDialog também é chamado, porém, em vez de abrir um 
arquivo, ele salva no diretório escolhido pelo usuário.

\subsubsection{Escolhendo shader}

Existe duas maneiras que o usuário pode escolher os shaders dentro da interface. A primeira é usando 
os keybinds no teclado número, apertando qualquer um dos números de um a quatro, faz com que o 
OpenGL atualize os shaders, atráves da função \emph{keyPressEvent ( QKeyEvent *event )}. Entretanto, 
o usuário também pode atualizar os shaders usando a QComboBox mencionada na seção sobre a aba de edição.
Neste caso, a função que será chamada é a \emph{changeShader(const QString shaderName)}, que irá trocar 
o shader de acordo com o nome enviado para ela.

\subsubsection{Movimento do mouse}

Além de fazer leitura do teclado, o programa também faz leitura do mouse, ele 
reconhece entradas no botão esquerdo do mouse (quando o objeto quer ser rotacionado) e também o scroll do mouse 
quando o usuário quer fazer uma operação de ampliação. As funções responsáveis para interagir com as entradas do mouse
são: \emph{GLWidget :: mouseMoveEvent (QMouseEvent *event)}, \emph{void GLWidget :: mousePressEvent (QMouseEvent *event)},
\emph{void GLWidget :: mouseReleaseEvent (QMouseEvent *event)}, \emph{void GLWidget :: wheelEvent (QWheelEvent *event)}.


\subsection{light.cpp}

Classe para gerenciar as luzes dentro do programa, o glwidget.cpp possui uma 
instância do objeto, que será usada durante o processo de "desenhar" a tela \emph{paintGL()}, para 
calcular a iluminação.

\subsection{material.cpp}

Usado em junção com o light.cpp, o Material é uma classe para descrever 
as propriedades de um objeto quando for refletido pela luz. Assim como ela, 
todos esses cálculos são feitos na hora de chamar a função \emph{paintGL()}.

\subsection{camera.cpp e trackball.cpp}

Por fim, para rotacionar e ampliar o objeto 3D, todos 
os cálculos são feitos em cima das classes Camera e Trackball. Essas classes 
também são chamadas quando o OpenGL altera o tamanho da janela da aplicação.


